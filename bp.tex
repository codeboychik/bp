\documentclass[12pt]{article}
\usepackage{xdipp} % Replace 'mystyle' with the name of your .sty file

\begin{document}

% Your document content goes here

\titul
{Implementace evidence docházky pro elektronický docházkový systém Mendelovy univerzity v Brně}
{Andrey Chernenko}
{Ing. Oldřich Faldík, Ph.D.}
{Brno 2023}

\podekovani{
        Chtěl bych poděkovat svému vedoucímu Oldřichu Faldíkovi, 
    Ph.D., za jeho vedení a rady při vypracování této bakalářské práce. Jeho odborné 
    znalosti a rady byly pro mě velkou pomocí. Děkuji také rodině a 
    přátelům za jejich podporu během mého studia.
}


\prohlasenimuz
{Brno 21.9.2023}

\abstract{}{
        This bachelor thesis deals with the implementation of a reports component, 
    which is a part of attendance records in the Smart Mendelu environment. 
    The thesis also contains a general overview of the implementation of microservices 
    and examines the benefits and shortcomings of using a microservices architecture. 
    The solution was developed using the object-oriented programming language Java 
    and its Spring Boot framework, along with JavaScript and its Vue.js framework 
    to create a reactive user interface.
}
\abstrakt{}{
        Tato bakalářská práce se zabývá implementací komponenty sestavy, 
    která je součástí evidence docházky v prostředí Smart Mendelu. 
    Práce dále obsahuje obecný přehled problematiky implementace mikroslužeb 
    a zkoumá přínosy a nedostatky použití architektury mikroslužeb. 
    Řešení bylo vytvořeno s využitím objektově orientovaného programovacího jazyka Java 
    a jeho frameworku Spring Boot, spolu s JavaScriptem a jeho frameworkem Vue.js 
    pro vytvoření reaktivního uživatelského rozhraní.
}

\obsah
\cislovat{1}
\cislovat{2}


\kapitola{Úvod}
\sekce{Úvod do práce}
    Evidence docházky je důležitou součástí správy lidských zdrojů ve většině organizací různé
    velikostí. Elektronické docházkové systémy se staly běžnou cestou pro správu docházky,
    protože poskytují vysokou úroveň automatizace a přesnosti. Volba určitých nástrojů a
    architektur pro sestavení a zprovoznění aplikaci má své výhody a nevýhody. 
\sekce{Cíl práce}
    Cílem této práce je definovat problémy spojené s podstatou implementace evidence docházky a 
    vymezit funkční a nefunkční požadavky, které by tyto problémy pokryly. Během implementace 
    využit vyhod mikroservisní architektury na backendové straně a architektury jednostrankových 
    aplikací na frontendové straně. Budou představeny různé metody a technologie, které se používají 
    při vývoji informačních systémů.
\kapitola{Současný stav evidence docházky}
\sekce{Architektura}
\sekce{Návrh}

\sekce{Bezpečnost}
\sekce{Monolitní aplikace}
    Komponenta ED, která je využívána v současné době v produkčním prostředí, je postavena
    na monolitické architektuře a napsána v skriptovacím jazyce PHP. Tento přístup zahrnuje:
    \begin{itemize}
        \item Jednu základnu kódu
        \item Jednu databázi pro všechna data
        \item Jeden nebo několik serveru, na kterých běži instance aplikace
        \item Jeden bod vstupu pro všechny požadavky (index.php)
    \end{itemize}

    Výhody tohoto prístupu jsou v tom, že za prvé je jednoduchá a pochopitelná na
    začátku vývoje. Za druhé je dobře škálovatelná v rámcí jednoho zařízení, protože základná
    kódu se nachází na jednom místě. Ze stejného důvodu je snadno jí nasazovat, testovat a je
    proto nákladově efektivní.
    Evidence docházky není jedinou komponentou, která běží na serveru, spolu s ní
    neoddělitelně běží i další částí aplikace. Během času kódová základna informačního systému
    docházky se bude rozšiřovat, částí kódu se stanou na sobě více závislé. Kvůli tomu ladění a
    oprava chyb, testování jednotlivých částí kódu, nasazení budou probíhat obtížněji, protože se
    zvyšuje riziko poškodit program na jiném místě. Při škálování monolitického systému také
    nastává problém protože pokud bude třeba oddělit a přenést na jiné zařízení určitou část
    systému, to prakticky nebude uskutečnitelné.
\sekce{Výhody použití nových technologií}
    Podle nefunkčních požadavku popsaných v předchozí kapitole vyhovující
    architekturou na straně serveru bude architektura mikroslužeb. Podle Newmana (2015) mikroslužby jsou
    přístupem distribučních systémů které propagují použití dobře rozdělených služeb se
    samostatnými životními cykly, pro vzájemnou spolupraci. Mikroslužby mají několik výhod
    oproti monolitické architektuře, mezi nimi patří:
\sekce{Současné trendy vývoje webových aplikací}
    (ne)kratký přehled současných architektur a principů postavení aplikací
\begin{itemize}
    \item Škálovatelnost, díky které častí aplikaci se dá spustit nezávislé a na různých
    zařízeních.
    \item Odolnost: pokud z určitých důvodů komponenta přestane fungovat, ostatní
    komponenty zůstanou pokračovat v běhu.
    \item Znovupoužitelnost: v případě nadměrné zátěže na určitou komponentu lze tuto
    komponentu duplikovat a provést load balancing.
    \item Testování: jednotlivé mikroslužby je možné testovat nezávisle na ostatních.
    \item Jednoduchost vnímání a pochopení programu: v menší komponentě lze snadněji
    sledovat závislostí kódu.
\end{itemize}

\kapitola{Postup implementace}
Popis principu, výhody/nevýhody
\sekce{Backendová část}
Popís struktury Spring Boot aplikace: endpointy, napojení na databazi, 
použite nastroje/baličky
\sekce{Frontendová část}
Popís struktury Vue.js aplikace: implementace vizualních komponent 
(tabulky, uživatelský panel), autentikace, použite nastroje/baličky. 
\newline
\newline
Poznámky:
\begin{itemize}
    \item Konfigurace adapteru pro identity management nástroj Keycloak, vlastní plugin Auth pro Vue
    \item Cross-site request forgery https://security.stackexchange.com/questions/20187/oauth2-cross-site-request-forgery-and-state-parameter
    \item Balik vue-router pro routing v SPA aplikacích, princip routování u spa aplikací
    \item State management plugin Vuex pro framework Vue

\end{itemize}

\end{document}
